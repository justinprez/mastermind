\documentclass[12pt, titlepage]{article}

\usepackage{graphicx}
\usepackage{paralist}
\usepackage{amsfonts}
\usepackage{amsmath}
\usepackage{hhline}
\usepackage{booktabs}
\usepackage{multirow}
\usepackage{multicol}
\usepackage{tabularx}
\usepackage[normalem]{ulem}
\usepackage{xcolor}
\usepackage{float}
\usepackage{hyperref}
\hypersetup{
    colorlinks=true,
    linkcolor=blue,
    filecolor=magenta,      
    urlcolor=cyan,
}

\title{SE 3XA3: Test Report\\Mastermind}

\author{Team 204, Trident Inc
		\\ Justin Prez, prezj
		\\ Justin Rosner, rosnej1
		\\ Harshil Modi, modih1
}

\date{\today}

%\input{../Comments}

\begin{document}

\maketitle

\pagenumbering{roman}
\tableofcontents
\listoftables
\listoffigures

\begin{table}[H]
\caption{\bf Revision History}
\begin{tabularx}{\textwidth}{p{3cm}p{2cm}X}
\toprule {\bf Date} & {\bf Version} & {\bf Notes}\\
\midrule
2020-04-05 & 1.0 & Initial version of Test Report\\
\bottomrule
\end{tabularx}
\end{table}

\newpage

\pagenumbering{arabic}

This document provides a report of the testing that was performed on the Mastermind App. The doc also provides changes performed as a result of testing, a traceabiliy matrix for the test results, and code converage metrics.

\section{Functional Requirements Evaluation}
\subsection{User Input}
\begin{itemize}
  \item Test Name: UI1 - Test selection of blue colour peg\\
        Result: This test was successful as as a blue peg was placed in the next available location upon clicking on the blue coloured peg from the list of pegs.
    \item Test Name: UI2 - Test selection of yellow colour peg\\
        Result: This test was successful as as a yellow peg was placed in the next available location upon clicking on the yellow coloured peg from the list of pegs.
    \item Test Name: UI3 - Test selection of red colour peg\\
        Result: This test was successful as as a red peg was placed in the next available location upon clicking on the red coloured peg from the list of pegs.
    \item Test Name: UI4 - Test selection of green colour peg\\
        Result: This test was successful as as a green peg was placed in the next available location upon clicking on the green coloured peg from the list of pegs.
    \item Test Name: UI5 - Test selection of white colour peg\\
        Result: This test was successful as as a white peg was placed in the next available location upon clicking on the white coloured peg from the list of pegs.
    \item Test Name: UI6 - Test selection of purple colour peg\\
        Result: This test was successful as as a purple peg was placed in the next available location upon clicking on the purple coloured peg from the list of pegs.
    \item Test name: UI7: Test simultaneous button presses \\
        Result: This test was successful as the device would not register the event when two buttons were pressed at the same time. No errors were generated.
    \item Test name: UI8: Test undo button \\
        Result: This test was successful as the user is able to undo their latest changes by pressing the undo button. The undo button also does not work if hints have been provided for a row. This is because then the user will be able to cheat the game.
    \item Test name: UI9: Test game instructions are displayed to the user upon start up of the game \\
        Result: This test was successful as the user is able to see the game instructions as soon as the game start or the user goes to the start screen.
    \item Test name: UI10: Test close game button \\
        Result: This test was successful as the game screen goes to the start menu upon clicking on the 'X' button in the top left corner of the game.
    \item Test name: UI11: Test new game button \\
        Result: This test was successful as upon clicking on the new game button the user is directed to the game screen.
\end{itemize}
\subsection{Game Environment}
\begin{itemize}
    \item Test name: GE1: Test load start menu upon startup \\
        Result: This test was successful. When the user click on the Mastermind app, the user is directed to the start menu.
    \item Test name: GE2: Test load help menu upon startup \\
        Result: This test was successful. When the user click on the Mastermind app, the user is directed to the start menu which also contains the help menu.
    \item Test name: GE3: Test load game board \\
        Result: This test was successful. When the user click on the play button the user the game board is loaded to the screen.
\end{itemize}
\subsection{Game Logic}
\begin{itemize}
    \item Test name: GL1: Test full peg row \\
        Result: This test was successful. When the user clicked on a next peg to be added, the next peg was added to the next row when the previous row contained 4 pegs already.
    \item Test name: GL2: Test incorrect guess (that doesn't finish the game) \\
        Result: This test was successful. When a row is full and the game has not been won, the hints are added to show the number of correct guess and the user is able to continue adding pegs to the next row.
    \item Test name: GL3: Test win game \\
        Result: This test was successful. When the user correctly guesses the combination, a win state message is displayed to the user and the user is directed to the main menu.
    \item Test name: GL4: Test lose game \\
        Result: This test was successful. When the user guesses 40 pegs and the game as not been won, the user gets a message saying that they lost the game and the correct combination is displayed to the user.
\end{itemize}

\section{Nonfunctional Requirements Evaluation}
\subsection{Look and Feel}
\begin{itemize}
        \item Test name: LF1: Test that the game board resembles the classic game board from the 70's \\
        Result: The testers believe that the overall feel of the game is a newer version of the 70's game with a wooden background.
\end{itemize}

\subsection{Usability and Humanity Requirements}
\begin{itemize}
    \item Test name: UHR1: Test that the game shall be playable by user MIN\_AGE and up \\
    Result: The developers asked the their younger sibling, parents and grandparents to follow the instructions and play the game. The tests was successful as everyone was able to play the game without any problems.
    \item Test name: UHR2: Test that the user will remember how to play the game \\
    Result: This test was successful. When the game was given to a player who played the game a week ago, the user was still able to remember how to play the game without any instructions. 
\end{itemize}

\subsection{Performance}
\begin{itemize}
    \item Test name: P1: Speed and latency \\
    Result: When carrying out the following test no visual latency was observed. Furthermore, Flutter build in features did not identity much latency either.
    \item Test name: P2: Precision and Accuracy \\
    Result: When playing the game multiple times and changing the options, the game performed as intended. No in-game glitches were experienced during the game.
    \item Test name: P3: System Stress Test \\
    Result: This test passed. When the tester played the game and spam clicked a peg the game was able to update faster then the user was able to click. 
\end{itemize}

\subsection{Operational and Environmental}
\begin{itemize}
    \item Test name: OE1: The game can be loaded on an Android system \\
          Result: The test passed as the game was successfully able to be downloaded on an Android device.
\end{itemize}
	
\section{Comparison to Existing Implementation}	
This section does not apply as tests were not provided in the original implementation and our test changes could not be tested on the original implementation.

\section{Unit Testing}
Unit testing was only performed on the game logic section as Unit testing the physical UI is not possible. Each function within the this section was tested using flutter test framework. Unit tests was performed for both returned value and exception handling. See \href{https://gitlab.cas.mcmaster.ca/rosnej1/open-mastermind/-/blob/master/Doc/Design/MIS/MIS.pdf}{MIS} for more information and \href{https://gitlab.cas.mcmaster.ca/rosnej1/open-mastermind/-/blob/master/src/mastermind_gui/test/unit_test.dart}{unit\_test.dart} for our unit testing.\\

All 4 of the unit tests passed. Within each unit test - multiple tests were performed to test both regular and boundary conditions. Overall, all the functions produced the correct output to the provided input.

\section{Changes Due to Testing}

\subsection{Functional Requirements Evaluation}

Few small changes were made to Mastermind due to tests corresponding to the functional requirements. With the recent revision of our test plan (Rev 1) we designed new functional tests for features of our game. Specifically, tests UI7 and UI8 were introduced to test the buttons handling multiple inputs and the functionality of the undo button. The UI7 test passed as the game did not error when simultaneously pressing the coloured buttons. UI8 tested for a feature not yet implemented, thus we had to adjust the Mastermind game board to accommodate for the new button. After adding the undo button, UI8 was tested and passed. All other functional tests for user input, menu navigation, game environment, and game logic passed.

\subsection{Look and Feel Requirements Evaluation}

Changes were made to the aesthetics of the menu and game board screen due to test LF1. Approximately 20 people were shown the initial demonstration of the game (Rev 0) and were to figure out how to play on their own, and comment on their experience using the app. Almost all users could play without any assistance, but many commented on the need for better application visuals. Our team redesigned the entire menu page, and change the background appearance of the application to resemble a fresh installment of the Mastermind from the 1970's.

\subsection{Usability and Humanity Evaluation}

There have been no changes made due to tests corresponding to the usability and humanity requirements test cases.

\subsection{Performance Evaluation}

There have been no changes made due to tests corresponding to the performance requirements test cases.

\subsection{Operation and Environmental Evaluation}

Changes were made to application to accommodate different screen sizes including phones and tablets. Although test OE1 does not specifically require us to use more than 1 type of emulator, we decided to experiment with emulators of different screen sizes to see how it would affect the appearance of our application. To ensure that the components of the application were not distorted we needed to align the components correctly so that they would appear in the correct position relative to each screen.


\section{Automated Testing}
Automated testing was performed throughout the development process to ensure that all versions of the mastermind worked as intended. Automated testing was done using Flutter's in-built feature which runs each unit test, widget test, and integration test when the "Run button" is pressed. Unit testing was done to insure that the game logic was not changed as in flutter the back-end and front-end code is intertwined. Widget testing was done to ensure that each widget was functional with its built in features. Finally, integrating testing was done to make sure that the system works when combined and that the app is functional on both ios and android.
		
\section{Trace to Requirements}
This section shows the Traceability Matrix between the test cases and their corresponding requirements.

\begin{table}[H]
\centering
\begin{tabular}{p{0.2\textwidth} p{0.6\textwidth}}
\toprule
\textbf{Test Case} & \textbf{Req.}\\
\midrule
UI1 & REQ2, REQ8\\
UI2 & REQ3, REQ8\\
UI3 & REQ4, REQ8\\
UI4 & REQ5, REQ8\\
UI5 & REQ6, REQ8\\
UI6 & REQ7, REQ8\\
UI7 & REQ8\\
UI8 & REQ14\\
UI9 & REQ9\\
UI10 & REQ10\\
UI11 & REQ9, OE4, OE2\\
GE1 & REQ9, OE4, OE2\\
GE2 & OE4, OE2\\
GE3 & OE4, OE2 \\
GL1 &  REQ11, REQ12, REQ8\\
GL2 &  REQ12, REQ8\\
GL3 & REQ10, REQ11, REQ12\\
GL4 &  REQ13\\
LF1 &  LF1\\
UHR1 &  UH1\\
UHR2 &  UH7\\
P1 &  PR1\\
P2 & PR3, PR4 \\
P3 & PR8, PR7, PR3\\
OE1 & REQ1, OE1, OE2\\

\bottomrule
\end{tabular}
\caption{Trace Between Test Cases and Requirements}
\label{TblRT}
\end{table}

		
\section{Trace to Modules}		This section shows the Traceability Matrix from the Test cases to the Modules outlined in the MIS and MG. For further reference to the modules please see the \href{https://gitlab.cas.mcmaster.ca/rosnej1/open-mastermind/-/blob/master/Doc/Design/MG/MG.pdf}{\textcolor{blue}{MG}} and \href{https://gitlab.cas.mcmaster.ca/rosnej1/open-mastermind/-/blob/master/Doc/Design/MIS/MIS.pdf}{\textcolor{blue}{MIS}} documents.

For reference the modules are as follows:
\begin{itemize}
    \item M1: Android Emulator
    \item M2: Game Board Module
    \item M3: Menu Module
    \item M4: Button Module
    \item M5: Game Board Controller
    \item M6: Types Module
\end{itemize}

\begin{table}[H]
\centering
\begin{tabular}{p{0.2\textwidth} p{0.6\textwidth}}
\toprule
\textbf{Test Case} & \textbf{Modules}\\
\midrule
UI1 & M2, M4, M6\\
UI2 & M2, M4, M6\\
UI3 & M2, M4, M6\\
UI4 & M2, M4, M6\\
UI5 & M2, M4, M6\\
UI6 & M2, M4, M6\\
UI7 & M4, M2, M5\\
UI8 & M2, M5\\
UI9 & M1, M3\\
UI10 & M2, M3\\
UI11 & M2, M3\\
GE1 & M1, M3\\
GE2 & M1, M3\\
GE3 &  M2\\
GL1 & M5\\
GL2 & M5\\
GL3 & M3, M5\\
GL4 &  M3, M5\\
LF1 &  M2\\
UHR1 &  M1, M2, M3\\
UHR2 &  M1, M2, M3\\
P1 &  M1\\
P2 &  M1, M5\\
P3 & M2, M4, M5, M6 \\
OE1 & M1\\

\bottomrule
\end{tabular}
\caption{Trace Between Test Cases and Requirements}
\end{table}

\section{Code Coverage Metrics}

Trident Inc. has managed to get 98\% statement and branch coverage through testing the Mastermind application. This number was generated by running the command `flutter test --coverage` in the command line in an existing flutter project with flutter tests and proper dependencies installed. Branch coverage is much more difficult to obtain than statement coverage due to the majority of the branches being dependent on a unique sequence of user inputs. The resulting code coverage metrics give the development team confidence in their testing suite, and that they have amply tested the implementation.

\end{document}
